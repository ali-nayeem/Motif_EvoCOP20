\section{Introduction}
\label{sec:intro}
Unveiling the mechanisms that control gene expression is a major challenge in biology. One of the most important tasks in this challenge is to identify \textbf{conserved regions} in deoxyribonucleic acid (DNA) for transcription factors. These \textit{short conserved regions} are known as motifs.

Recent advancement in genome sequencing technology has made it possible to do sequencing of whole genome of different organisms. Also the advancement in gene expression analysis has allowed for the development of many computational methods for motif finding.

There are basically two types of motif discovery algorthms; i.e. enumeration approach and probabilistic technique\cite{hashim2019review}. Enumeration Approach algorithms search for patterns present in each sequence. Motifs are predicted based on frequency of words and hamming distance between words. So, these algorithms are sometimes called Panted (l, d) Motif Problem (PMP), where motif length $= l$ and a maximum number of mismatches $= d$. Popular algorithms based on this approach are Weeder \cite{zhang2016entropy}, CisFinder\cite{bailey2011dreme}, MCES\cite{jia2014new}.

Probabilistic approaches construct Position-Specific Weight Matrix (PSWM) that specifies a distribution of bases for each position in Transcription Factor Binding SItes to distinguish motifs vs. non-motifs\cite{hashim2019review}. The most popular algorithms based on this method are MEME\cite{sinha2003ymf}, EXTREME\cite{thomas2011rsat}, DABC\cite{karaboga2019discovery} and BioProspector\cite{pavesi2001algorithm}.

There exists \textbf{Multi-Objetive Optimization} algorithms which try to optimize conflicting objectives to better estimate the motifs. Mo-ABC/DE algorithm combines Artificial Bee Colony Algorithm(ABC) with Differential Evolution(DE) to optimize three conflicting objectives\cite{gonzalez2013hybrid}. In that paper, they compared their performance with Multiobjective Artificial Bee Colony (MOABC, \cite{gonzalez2013comparing}), Differential Evolution with Pareto Tournaments (DEPT,\cite{gonzalez2011predicting}), Non-dominated Sorting Genetic Algorithm II (NSGA-II,\cite{deb2002fast}) and Strength Pareto Evolutionary Algorithm 2 (SPEA2,\cite{zitzler2000improving}). They used \textbf{hypervolume} as their performance metric.

We believe hypervolume is not the appropriate performance measure for practical problems like \texit{motif finding}. Hypervolume is used to compare algorithms to see which algorithm is covering more of the search space. In this paper, we propose an NSGAII algorithm with two objectives and compare our algorithm's performance with state of the art predictors. We also observe how our hypervolume differs with other multi-objective optimization algorithms.

